% Options for packages loaded elsewhere
\PassOptionsToPackage{unicode}{hyperref}
\PassOptionsToPackage{hyphens}{url}
%
\documentclass[
]{article}
\usepackage{amsmath,amssymb}
\usepackage{iftex}
\ifPDFTeX
  \usepackage[T1]{fontenc}
  \usepackage[utf8]{inputenc}
  \usepackage{textcomp} % provide euro and other symbols
\else % if luatex or xetex
  \usepackage{unicode-math} % this also loads fontspec
  \defaultfontfeatures{Scale=MatchLowercase}
  \defaultfontfeatures[\rmfamily]{Ligatures=TeX,Scale=1}
\fi
\usepackage{lmodern}
\ifPDFTeX\else
  % xetex/luatex font selection
\fi
% Use upquote if available, for straight quotes in verbatim environments
\IfFileExists{upquote.sty}{\usepackage{upquote}}{}
\IfFileExists{microtype.sty}{% use microtype if available
  \usepackage[]{microtype}
  \UseMicrotypeSet[protrusion]{basicmath} % disable protrusion for tt fonts
}{}
\makeatletter
\@ifundefined{KOMAClassName}{% if non-KOMA class
  \IfFileExists{parskip.sty}{%
    \usepackage{parskip}
  }{% else
    \setlength{\parindent}{0pt}
    \setlength{\parskip}{6pt plus 2pt minus 1pt}}
}{% if KOMA class
  \KOMAoptions{parskip=half}}
\makeatother
\usepackage{xcolor}
\usepackage[margin=1in]{geometry}
\usepackage{color}
\usepackage{fancyvrb}
\newcommand{\VerbBar}{|}
\newcommand{\VERB}{\Verb[commandchars=\\\{\}]}
\DefineVerbatimEnvironment{Highlighting}{Verbatim}{commandchars=\\\{\}}
% Add ',fontsize=\small' for more characters per line
\usepackage{framed}
\definecolor{shadecolor}{RGB}{248,248,248}
\newenvironment{Shaded}{\begin{snugshade}}{\end{snugshade}}
\newcommand{\AlertTok}[1]{\textcolor[rgb]{0.94,0.16,0.16}{#1}}
\newcommand{\AnnotationTok}[1]{\textcolor[rgb]{0.56,0.35,0.01}{\textbf{\textit{#1}}}}
\newcommand{\AttributeTok}[1]{\textcolor[rgb]{0.13,0.29,0.53}{#1}}
\newcommand{\BaseNTok}[1]{\textcolor[rgb]{0.00,0.00,0.81}{#1}}
\newcommand{\BuiltInTok}[1]{#1}
\newcommand{\CharTok}[1]{\textcolor[rgb]{0.31,0.60,0.02}{#1}}
\newcommand{\CommentTok}[1]{\textcolor[rgb]{0.56,0.35,0.01}{\textit{#1}}}
\newcommand{\CommentVarTok}[1]{\textcolor[rgb]{0.56,0.35,0.01}{\textbf{\textit{#1}}}}
\newcommand{\ConstantTok}[1]{\textcolor[rgb]{0.56,0.35,0.01}{#1}}
\newcommand{\ControlFlowTok}[1]{\textcolor[rgb]{0.13,0.29,0.53}{\textbf{#1}}}
\newcommand{\DataTypeTok}[1]{\textcolor[rgb]{0.13,0.29,0.53}{#1}}
\newcommand{\DecValTok}[1]{\textcolor[rgb]{0.00,0.00,0.81}{#1}}
\newcommand{\DocumentationTok}[1]{\textcolor[rgb]{0.56,0.35,0.01}{\textbf{\textit{#1}}}}
\newcommand{\ErrorTok}[1]{\textcolor[rgb]{0.64,0.00,0.00}{\textbf{#1}}}
\newcommand{\ExtensionTok}[1]{#1}
\newcommand{\FloatTok}[1]{\textcolor[rgb]{0.00,0.00,0.81}{#1}}
\newcommand{\FunctionTok}[1]{\textcolor[rgb]{0.13,0.29,0.53}{\textbf{#1}}}
\newcommand{\ImportTok}[1]{#1}
\newcommand{\InformationTok}[1]{\textcolor[rgb]{0.56,0.35,0.01}{\textbf{\textit{#1}}}}
\newcommand{\KeywordTok}[1]{\textcolor[rgb]{0.13,0.29,0.53}{\textbf{#1}}}
\newcommand{\NormalTok}[1]{#1}
\newcommand{\OperatorTok}[1]{\textcolor[rgb]{0.81,0.36,0.00}{\textbf{#1}}}
\newcommand{\OtherTok}[1]{\textcolor[rgb]{0.56,0.35,0.01}{#1}}
\newcommand{\PreprocessorTok}[1]{\textcolor[rgb]{0.56,0.35,0.01}{\textit{#1}}}
\newcommand{\RegionMarkerTok}[1]{#1}
\newcommand{\SpecialCharTok}[1]{\textcolor[rgb]{0.81,0.36,0.00}{\textbf{#1}}}
\newcommand{\SpecialStringTok}[1]{\textcolor[rgb]{0.31,0.60,0.02}{#1}}
\newcommand{\StringTok}[1]{\textcolor[rgb]{0.31,0.60,0.02}{#1}}
\newcommand{\VariableTok}[1]{\textcolor[rgb]{0.00,0.00,0.00}{#1}}
\newcommand{\VerbatimStringTok}[1]{\textcolor[rgb]{0.31,0.60,0.02}{#1}}
\newcommand{\WarningTok}[1]{\textcolor[rgb]{0.56,0.35,0.01}{\textbf{\textit{#1}}}}
\usepackage{graphicx}
\makeatletter
\def\maxwidth{\ifdim\Gin@nat@width>\linewidth\linewidth\else\Gin@nat@width\fi}
\def\maxheight{\ifdim\Gin@nat@height>\textheight\textheight\else\Gin@nat@height\fi}
\makeatother
% Scale images if necessary, so that they will not overflow the page
% margins by default, and it is still possible to overwrite the defaults
% using explicit options in \includegraphics[width, height, ...]{}
\setkeys{Gin}{width=\maxwidth,height=\maxheight,keepaspectratio}
% Set default figure placement to htbp
\makeatletter
\def\fps@figure{htbp}
\makeatother
\setlength{\emergencystretch}{3em} % prevent overfull lines
\providecommand{\tightlist}{%
  \setlength{\itemsep}{0pt}\setlength{\parskip}{0pt}}
\setcounter{secnumdepth}{-\maxdimen} % remove section numbering
\ifLuaTeX
  \usepackage{selnolig}  % disable illegal ligatures
\fi
\usepackage{bookmark}
\IfFileExists{xurl.sty}{\usepackage{xurl}}{} % add URL line breaks if available
\urlstyle{same}
\hypersetup{
  pdftitle={Assignment 1: California Spiny Lobster Abundance (Panulirus Interruptus)},
  pdfauthor={EDS 241},
  hidelinks,
  pdfcreator={LaTeX via pandoc}}

\title{Assignment 1: California Spiny Lobster Abundance (\emph{Panulirus
Interruptus})}
\usepackage{etoolbox}
\makeatletter
\providecommand{\subtitle}[1]{% add subtitle to \maketitle
  \apptocmd{\@title}{\par {\large #1 \par}}{}{}
}
\makeatother
\subtitle{Assessing the Impact of Marine Protected Areas (MPAs) at 5
Reef Sites in Santa Barbara County}
\author{EDS 241}
\date{1/8/2024 (Due 1/22)}

\begin{document}
\maketitle

\section{Assignment submission: Nicole
Pepper}\label{assignment-submission-nicole-pepper}

\begin{center}\rule{0.5\linewidth}{0.5pt}\end{center}

\includegraphics{figures/spiny2.jpg}

\begin{center}\rule{0.5\linewidth}{0.5pt}\end{center}

\subsubsection{Assignment instructions:}\label{assignment-instructions}

\begin{itemize}
\item
  Working with partners to troubleshoot code and concepts is encouraged!
  If you work with a partner, please list their name next to yours at
  the top of your assignment so Annie and I can easily see who
  collaborated.
\item
  All written responses must be written independently (\textbf{in your
  own words}).
\item
  Please follow the question prompts carefully and include only the
  information each question asks in your submitted responses.
\item
  Submit both your knitted document and the associated
  \texttt{RMarkdown} or \texttt{Quarto} file.
\item
  Your knitted presentation should meet the quality you'd submit to
  research colleagues or feel confident sharing publicly. Refer to the
  rubric for details about presentation standards.
\end{itemize}

\textbf{Assignment submission (YOUR NAME):}
\_\_\_\_\_\_\_\_\_\_\_\_\_\_\_\_\_\_\_\_\_\_\_\_\_\_\_\_\_\_\_\_\_\_\_\_\_\_

\begin{center}\rule{0.5\linewidth}{0.5pt}\end{center}

\begin{Shaded}
\begin{Highlighting}[]
\CommentTok{\# {-}{-}{-}{-} load libraries {-}{-}{-}{-}}
\FunctionTok{library}\NormalTok{(tidyverse)}
\FunctionTok{library}\NormalTok{(here)}
\FunctionTok{library}\NormalTok{(janitor)}
\FunctionTok{library}\NormalTok{(estimatr)  }
\FunctionTok{library}\NormalTok{(performance)}
\FunctionTok{library}\NormalTok{(jtools)}
\FunctionTok{library}\NormalTok{(gt)}
\FunctionTok{library}\NormalTok{(gtsummary)}
\FunctionTok{library}\NormalTok{(MASS) }\DocumentationTok{\#\# }\AlertTok{NOTE}\DocumentationTok{: The \textasciigrave{}select()\textasciigrave{} function is masked. Use: \textasciigrave{}dplyr::select()\textasciigrave{} \#\#}
\FunctionTok{library}\NormalTok{(interactions) }
\end{Highlighting}
\end{Shaded}

\begin{center}\rule{0.5\linewidth}{0.5pt}\end{center}

\paragraph{DATA SOURCE:}\label{data-source}

Reed D. 2019. SBC LTER: Reef: Abundance, size and fishing effort for
California Spiny Lobster (Panulirus interruptus), ongoing since 2012.
Environmental Data Initiative.
\url{https://doi.org/10.6073/pasta/a593a675d644fdefb736750b291579a0}.
Dataset accessed 11/17/2019.

\begin{center}\rule{0.5\linewidth}{0.5pt}\end{center}

\subsubsection{\texorpdfstring{\textbf{Introduction}}{Introduction}}\label{introduction}

You're about to dive into some deep data collected from five reef sites
in Santa Barbara County, all about the abundance of California spiny
lobsters! 🦞 Data was gathered by divers annually from 2012 to 2018
across Naples, Mohawk, Isla Vista, Carpinteria, and Arroyo Quemado
reefs.

Why lobsters? Well, this sample provides an opportunity to evaluate the
impact of Marine Protected Areas (MPAs) established on January 1, 2012
(Reed, 2019). Of these five reefs, Naples, and Isla Vista are MPAs,
while the other three are not protected (non-MPAs). Comparing lobster
health between these protected and non-protected areas gives us the
chance to study how commercial and recreational fishing might impact
these ecosystems.

We will consider the MPA sites the \texttt{treatment} group and use
regression methods to explore whether protecting these reefs really
makes a difference compared to non-MPA sites (our control group). In
this assignment, we'll think deeply about which causal inference
assumptions hold up under the research design and identify where they
fall short.

Let's break it down step by step and see what the data reveals! 📊

\includegraphics{figures/map-5reefs.png}

\begin{center}\rule{0.5\linewidth}{0.5pt}\end{center}

Step 1: Anticipating potential sources of selection bias

\textbf{a.} Do the control sites (Arroyo Quemado, Carpinteria, and
Mohawk) provide a strong counterfactual for our treatment sites (Naples,
Isla Vista)? Write a paragraph making a case for why this comparison is
centris paribus or whether selection bias is likely (be specific!).

\emph{There are a lot of location-specific variables in marine
ecosystems that can't be fully controlled for, so I think that selection
bias is likely to be present not only between the control and treatment
sites, but also within the treatment and control groups themselves. From
what I found on
\href{https://wildlife.ca.gov/Conservation/Marine/MPAs/Naples}{wildlife.ca},
the two treatment sites have different fishing regulations; while Isla
Vista is a ``No-Take'' zone, Naples allows recreational take of some
pelagic fish and for the commercial take of kelp. I imagine that the
difference in fishing behavior could likely influence the broader marine
ecosystem, including for lobsters, which could introduce bias within the
treatment group. Additionally, the sites, though all located in the
greater Santa Barbara region, are dispersed broadly across the coast and
likely have different baselines in variables such as proximity to
pollution, habitat disturbance, among other environmental/human factors
that may introduce challenges in comparing the sites.}

\begin{center}\rule{0.5\linewidth}{0.5pt}\end{center}

Step 2: Read \& wrangle data

\textbf{a.} Read in the raw data. Name the data.frame (\texttt{df})
\texttt{rawdata}

\textbf{b.} Use the function \texttt{clean\_names()} from the
\texttt{janitor} package

\begin{Shaded}
\begin{Highlighting}[]
\CommentTok{\# {-}{-}{-}{-} read in project data {-}{-}{-}{-}}
\NormalTok{rawdata }\OtherTok{\textless{}{-}} \FunctionTok{read.csv}\NormalTok{(here}\SpecialCharTok{::}\FunctionTok{here}\NormalTok{(}\StringTok{"data"}\NormalTok{,}\StringTok{"spiny\_abundance\_sb\_18.csv"}\NormalTok{), }
                    \AttributeTok{na.strings =} \StringTok{"{-}99999"}\NormalTok{) }\SpecialCharTok{|\textgreater{}} \CommentTok{\# set to NA}
    \FunctionTok{clean\_names}\NormalTok{() }\CommentTok{\# clean column names}
\end{Highlighting}
\end{Shaded}

\textbf{c.} Create a new \texttt{df} named \texttt{tidydata}. Using the
variable \texttt{site} (reef location) create a new variable
\texttt{reef} as a \texttt{factor} and add the following labels in the
order listed (i.e., re-order the \texttt{levels}):

\begin{verbatim}
"Arroyo Quemado", "Carpinteria", "Mohawk", "Isla Vista",  "Naples"
\end{verbatim}

\begin{Shaded}
\begin{Highlighting}[]
\CommentTok{\# {-}{-}{-}{-} clean \& prep data {-}{-}{-}{-}}

\CommentTok{\# create labels for full site name and set to ordered factor}
\NormalTok{tidydata }\OtherTok{\textless{}{-}}\NormalTok{ rawdata }\SpecialCharTok{|\textgreater{}}
    \FunctionTok{mutate}\NormalTok{(}\AttributeTok{reef =} \FunctionTok{factor}\NormalTok{( site, }
                          \AttributeTok{levels =} \FunctionTok{c}\NormalTok{(}\StringTok{"AQUE"}\NormalTok{, }\StringTok{"CARP"}\NormalTok{, }\StringTok{"MOHK"}\NormalTok{, }\StringTok{"IVEE"}\NormalTok{,  }\StringTok{"NAPL"}\NormalTok{),}
                          \AttributeTok{labels =} \FunctionTok{c}\NormalTok{(}\StringTok{"Arroyo Quemado"}\NormalTok{, }\StringTok{"Carpinteria"}\NormalTok{, }\StringTok{"Mohawk"}\NormalTok{, }\StringTok{"Isla Vista"}\NormalTok{,  }\StringTok{"Naples"}\NormalTok{)}
\NormalTok{    ))}
\end{Highlighting}
\end{Shaded}

Create new \texttt{df} named \texttt{spiny\_counts}

\textbf{d.} Create a new variable \texttt{counts} to allow for an
analysis of lobster counts where the unit-level of observation is the
total number of observed lobsters per \texttt{site}, \texttt{year} and
\texttt{transect}.

\begin{itemize}
\tightlist
\item
  Create a variable \texttt{mean\_size} from the variable
  \texttt{size\_mm}
\item
  NOTE: The variable \texttt{counts} should have values which are
  integers (whole numbers).
\item
  Make sure to account for missing cases (\texttt{na})!
\end{itemize}

\textbf{e.} Create a new variable \texttt{mpa} with levels \texttt{MPA}
and \texttt{non\_MPA}. For our regression analysis create a numerical
variable \texttt{treat} where MPA sites are coded \texttt{1} and
non\_MPA sites are coded \texttt{0}

\begin{Shaded}
\begin{Highlighting}[]
\CommentTok{\# {-}{-}{-}{-} summarise and prepare lobster data for regression analysis {-}{-}{-}{-}}

\NormalTok{spiny\_counts }\OtherTok{\textless{}{-}}\NormalTok{ tidydata }\SpecialCharTok{|\textgreater{}}
    
    \CommentTok{\# group lobster data by site, reef, year, \& transect}
    \FunctionTok{group\_by}\NormalTok{(site, year, transect) }\SpecialCharTok{|\textgreater{}}
    
    \CommentTok{\# count lobsters observed at each site{-}year{-}transect observation \& mean size}
    \FunctionTok{summarize}\NormalTok{(}
        \AttributeTok{counts =} \FunctionTok{sum}\NormalTok{(count, }\AttributeTok{na.rm =} \ConstantTok{TRUE}\NormalTok{),}
        \AttributeTok{mean\_size =} \FunctionTok{mean}\NormalTok{(size\_mm, }\AttributeTok{na.rm =} \ConstantTok{TRUE}\NormalTok{)}
\NormalTok{    ) }\SpecialCharTok{|\textgreater{}}
    \FunctionTok{ungroup}\NormalTok{() }\SpecialCharTok{|\textgreater{}}
    
    \CommentTok{\# create variables to distinguish MPA vs non{-}MPA sites}
    \FunctionTok{mutate}\NormalTok{(}
        \AttributeTok{mpa =} \FunctionTok{case\_when}\NormalTok{(}
\NormalTok{            site }\SpecialCharTok{\%in\%} \FunctionTok{c}\NormalTok{( }\StringTok{"IVEE"}\NormalTok{,  }\StringTok{"NAPL"}\NormalTok{) }\SpecialCharTok{\textasciitilde{}} \StringTok{"MPA"}\NormalTok{,}
            \ConstantTok{TRUE} \SpecialCharTok{\textasciitilde{}} \StringTok{"non\_MPA"}
\NormalTok{            ),}
        \AttributeTok{treat =} \FunctionTok{if\_else}\NormalTok{(mpa }\SpecialCharTok{==} \StringTok{"non\_MPA"}\NormalTok{, }\DecValTok{0}\NormalTok{,}\DecValTok{1}\NormalTok{)}
\NormalTok{    )}
\end{Highlighting}
\end{Shaded}

\begin{quote}
NOTE: This step is crucial to the analysis. Check with a friend or come
to TA/instructor office hours to make sure the counts are coded
correctly!
\end{quote}

\begin{center}\rule{0.5\linewidth}{0.5pt}\end{center}

Step 3: Explore \& visualize data

\textbf{a.} Take a look at the data! Get familiar with the data in each
\texttt{df} format (\texttt{tidydata}, \texttt{spiny\_counts})

\begin{Shaded}
\begin{Highlighting}[]
\CommentTok{\# {-}{-}{-}{-} explore data {-}{-}{-}{-}}

\FunctionTok{head}\NormalTok{(spiny\_counts)}
\FunctionTok{head}\NormalTok{(tidydata)}

\FunctionTok{dim}\NormalTok{(spiny\_counts)}
\FunctionTok{dim}\NormalTok{(tidydata)}
\end{Highlighting}
\end{Shaded}

\textbf{b.} We will focus on the variables \texttt{count},
\texttt{year}, \texttt{site}, and \texttt{treat}(\texttt{mpa}) to model
lobster abundance. Create the following 4 plots using a different method
each time from the 6 options provided. Add a layer (\texttt{geom}) to
each of the plots including informative descriptive statistics (you
choose; e.g., mean, median, SD, quartiles, range). Make sure each plot
dimension is clearly labeled (e.g., axes, groups).

\begin{itemize}
\tightlist
\item
  \href{https://r-charts.com/distribution/density-plot-group-ggplot2}{Density
  plot}
\item
  \href{https://r-charts.com/distribution/ggridges/}{Ridge plot}
\item
  \href{https://ggplot2.tidyverse.org/reference/geom_jitter.html}{Jitter
  plot}
\item
  \href{https://r-charts.com/distribution/violin-plot-group-ggplot2}{Violin
  plot}
\item
  \href{https://r-charts.com/distribution/histogram-density-ggplot2/}{Histogram}
\item
  \href{https://r-charts.com/distribution/beeswarm/}{Beeswarm}
\end{itemize}

Create plots displaying the distribution of lobster \textbf{counts}:

\begin{enumerate}
\def\labelenumi{\arabic{enumi})}
\tightlist
\item
  grouped by reef site\\
\item
  grouped by MPA status
\item
  grouped by year
\end{enumerate}

\begin{Shaded}
\begin{Highlighting}[]
\CommentTok{\# {-}{-}{-}{-} Grouped by reef site {-}{-}{-}{-}}

\CommentTok{\# Create a violin plot of lobster counts by reef site}
\NormalTok{spiny\_counts }\SpecialCharTok{|\textgreater{}} 
\FunctionTok{ggplot}\NormalTok{(}\FunctionTok{aes}\NormalTok{(}\AttributeTok{x =}\NormalTok{ counts, }\AttributeTok{y =}\NormalTok{ site)) }\SpecialCharTok{+}
  \FunctionTok{geom\_violin}\NormalTok{(}\AttributeTok{fill =} \StringTok{"cornflowerblue"}\NormalTok{) }\SpecialCharTok{+}
    \CommentTok{\# Add box plot to display quartiles}
    \FunctionTok{geom\_boxplot}\NormalTok{(}\AttributeTok{alpha =} \DecValTok{0}\NormalTok{) }\SpecialCharTok{+}
    \CommentTok{\# Add point for mean value}
    \FunctionTok{stat\_summary}\NormalTok{(}\FunctionTok{aes}\NormalTok{(}\AttributeTok{x =}\NormalTok{ counts, }\AttributeTok{y =}\NormalTok{ site),}
                 \AttributeTok{fun =} \StringTok{"mean"}\NormalTok{,}
                 \AttributeTok{shape =} \DecValTok{3}\NormalTok{) }\SpecialCharTok{+} 
    \FunctionTok{labs}\NormalTok{(}\AttributeTok{title =} \StringTok{"Lobster Counts at Study Sites"}\NormalTok{,}
         \AttributeTok{subtitle =} \StringTok{"+ indicates the mean count"}\NormalTok{,}
         \AttributeTok{x =} \StringTok{"Lobster Count"}\NormalTok{,}
         \AttributeTok{y =} \StringTok{"Site Name"}\NormalTok{)}
\end{Highlighting}
\end{Shaded}

\begin{Shaded}
\begin{Highlighting}[]
\CommentTok{\# {-}{-}{-}{-} Grouped by year {-}{-}{-}{-}}

\CommentTok{\# Create a bar chart of lobster counts by year}
\NormalTok{spiny\_counts }\SpecialCharTok{|\textgreater{}}
\FunctionTok{ggplot}\NormalTok{(}\FunctionTok{aes}\NormalTok{(}\AttributeTok{x =}\NormalTok{ year, }\AttributeTok{y =}\NormalTok{ counts, }\AttributeTok{fill =}\NormalTok{ mpa)) }\SpecialCharTok{+} 
    \FunctionTok{geom\_col}\NormalTok{(}\AttributeTok{position =} \StringTok{"dodge"}\NormalTok{) }\SpecialCharTok{+}
    
    \CommentTok{\# Add point for median}
    \FunctionTok{stat\_summary}\NormalTok{(}\FunctionTok{aes}\NormalTok{(}\AttributeTok{group =}\NormalTok{ mpa),}
                 \AttributeTok{fun =} \StringTok{"median"}\NormalTok{,}
                 \AttributeTok{shape =} \DecValTok{4}\NormalTok{,}
                 \AttributeTok{position =} \FunctionTok{position\_dodge}\NormalTok{(}\AttributeTok{width =} \FloatTok{0.95}\NormalTok{)) }\SpecialCharTok{+}
    \FunctionTok{labs}\NormalTok{(}\AttributeTok{title =} \StringTok{"Lobster Counts by Year by MPA Status"}\NormalTok{,}
         \AttributeTok{subtitle =} \StringTok{"X indicates the median count"}\NormalTok{,}
         \AttributeTok{x =} \StringTok{"Year"}\NormalTok{,}
         \AttributeTok{y =} \StringTok{"Count"}\NormalTok{,}
         \AttributeTok{fill =} \StringTok{"MPA Status"}\NormalTok{) }
\end{Highlighting}
\end{Shaded}

\begin{Shaded}
\begin{Highlighting}[]
\CommentTok{\# {-}{-}{-}{-} Grouped by mpa status {-}{-}{-}{-}}

\CommentTok{\# Create a jitter plot of counts by reef, grouped by mpa status}
\NormalTok{spiny\_counts }\SpecialCharTok{|\textgreater{}}
\FunctionTok{ggplot}\NormalTok{(}\FunctionTok{aes}\NormalTok{(}\AttributeTok{x =}\NormalTok{ counts, }\AttributeTok{y =}\NormalTok{ mpa, }\AttributeTok{color =}\NormalTok{ mpa)) }\SpecialCharTok{+}
  \FunctionTok{geom\_jitter}\NormalTok{(}\AttributeTok{alpha =} \FloatTok{0.5}\NormalTok{) }\SpecialCharTok{+} 
    
    \CommentTok{\#add boxplot to display quartiles}
    \FunctionTok{geom\_boxplot}\NormalTok{(}\AttributeTok{alpha =} \DecValTok{0}\NormalTok{, }\AttributeTok{color =} \StringTok{"black"}\NormalTok{, }\AttributeTok{size =}\NormalTok{ .}\DecValTok{65}\NormalTok{) }\SpecialCharTok{+}
    
    \CommentTok{\# Add point for median value}
    \FunctionTok{stat\_summary}\NormalTok{(}\FunctionTok{aes}\NormalTok{(}\AttributeTok{x =}\NormalTok{ counts, }\AttributeTok{y =}\NormalTok{ mpa),}
                 \AttributeTok{fun =} \StringTok{"mean"}\NormalTok{,}
                 \AttributeTok{shape =} \DecValTok{3}\NormalTok{,}
                 \AttributeTok{color =} \StringTok{"black"}\NormalTok{) }\SpecialCharTok{+}
    
    \FunctionTok{labs}\NormalTok{(}\AttributeTok{title =} \StringTok{"Lobster Counts by Site \& MPA Status"}\NormalTok{,}
         \AttributeTok{subtitle =} \StringTok{"+ indicates the mean count"}\NormalTok{,}
         \AttributeTok{x =} \StringTok{"Year"}\NormalTok{,}
         \AttributeTok{y =} \StringTok{"Count"}\NormalTok{,}
         \AttributeTok{color =} \StringTok{"MPA Status"}\NormalTok{)}
\end{Highlighting}
\end{Shaded}

Create a plot of lobster \textbf{size} :

\begin{enumerate}
\def\labelenumi{\arabic{enumi})}
\setcounter{enumi}{3}
\tightlist
\item
  You choose the grouping variable(s)!
\end{enumerate}

\begin{Shaded}
\begin{Highlighting}[]
\CommentTok{\# {-}{-}{-}{-} Create a plot of mean lobster size {-}{-}{-}{-}}

\NormalTok{spiny\_counts }\SpecialCharTok{|\textgreater{}}
    \FunctionTok{ggplot}\NormalTok{(}\FunctionTok{aes}\NormalTok{(}\AttributeTok{x =}\NormalTok{ mean\_size, }\AttributeTok{fill =}\NormalTok{ mpa)) }\SpecialCharTok{+}
    \FunctionTok{geom\_density}\NormalTok{(}\AttributeTok{alpha =} \FloatTok{0.8}\NormalTok{) }\SpecialCharTok{+} 
    
    \CommentTok{\# Add vertical line for median count for MPA group}
     \FunctionTok{geom\_vline}\NormalTok{(}\AttributeTok{data =}\NormalTok{ spiny\_counts }\SpecialCharTok{|\textgreater{}} \FunctionTok{filter}\NormalTok{(mpa }\SpecialCharTok{==} \StringTok{"MPA"}\NormalTok{),}
                \FunctionTok{aes}\NormalTok{(}\AttributeTok{xintercept =} \FunctionTok{median}\NormalTok{(mean\_size, }\AttributeTok{na.rm =} \ConstantTok{TRUE}\NormalTok{)), }
             \AttributeTok{color =} \StringTok{"\#B34640"}\NormalTok{, }
             \AttributeTok{linetype =} \StringTok{"dashed"}\NormalTok{, }
             \AttributeTok{size =} \DecValTok{1}\NormalTok{) }\SpecialCharTok{+}
    
    \CommentTok{\# Add vertical line for non{-}MPA group}
    \FunctionTok{geom\_vline}\NormalTok{(}\AttributeTok{data =}\NormalTok{ spiny\_counts }\SpecialCharTok{|\textgreater{}} \FunctionTok{filter}\NormalTok{(mpa }\SpecialCharTok{!=} \StringTok{"MPA"}\NormalTok{), }\FunctionTok{aes}\NormalTok{(}\AttributeTok{xintercept =} \FunctionTok{median}\NormalTok{(mean\_size, }\AttributeTok{na.rm =} \ConstantTok{TRUE}\NormalTok{)), }
             \AttributeTok{color =} \StringTok{"\#235959"}\NormalTok{, }
             \AttributeTok{linetype =} \StringTok{"dashed"}\NormalTok{, }
             \AttributeTok{size =} \DecValTok{1}\NormalTok{) }\SpecialCharTok{+}
    
    \FunctionTok{labs}\NormalTok{(}\AttributeTok{title =} \StringTok{"Distribution of Lobster Size by MPA Status"}\NormalTok{,}
         \AttributeTok{x =} \StringTok{"Mean Lobster Size (mm)"}\NormalTok{,}
         \AttributeTok{y =} \StringTok{"Density"}\NormalTok{,}
         \AttributeTok{fill =} \StringTok{"MPA Status"}\NormalTok{) }
\end{Highlighting}
\end{Shaded}

\textbf{c.} Compare means of the outcome by treatment group. Using the
\texttt{tbl\_summary()} function from the package
\href{https://www.danieldsjoberg.com/gtsummary/articles/tbl_summary.html}{\texttt{gt\_summary}}

\begin{Shaded}
\begin{Highlighting}[]
\CommentTok{\# Compare means of the outcome by treatment group}

\NormalTok{spiny\_counts }\SpecialCharTok{|\textgreater{}} 
    \FunctionTok{ungroup}\NormalTok{() }\SpecialCharTok{|\textgreater{}}
\NormalTok{    dplyr}\SpecialCharTok{::}\FunctionTok{select}\NormalTok{(treat, counts) }\SpecialCharTok{|\textgreater{}}
    \FunctionTok{tbl\_summary}\NormalTok{(}
        \AttributeTok{by =}\NormalTok{ treat,}
        \AttributeTok{statistic =} \FunctionTok{list}\NormalTok{(}\FunctionTok{all\_continuous}\NormalTok{() }\SpecialCharTok{\textasciitilde{}} \StringTok{"\{mean\} (\{sd\})"}\NormalTok{))}\SpecialCharTok{|\textgreater{}}
    \FunctionTok{modify\_header}\NormalTok{(label }\SpecialCharTok{\textasciitilde{}} \StringTok{"Variable"}\NormalTok{)}\SpecialCharTok{|\textgreater{}}
    \FunctionTok{modify\_spanning\_header}\NormalTok{(}\FunctionTok{c}\NormalTok{(}\StringTok{"stat\_1"}\NormalTok{, }\StringTok{"stat\_2"}\NormalTok{) }\SpecialCharTok{\textasciitilde{}} \StringTok{"**Treatment**"}\NormalTok{)}
\end{Highlighting}
\end{Shaded}

\begin{center}\rule{0.5\linewidth}{0.5pt}\end{center}

Step 4: OLS regression- building intuition

\textbf{a.} Start with a simple OLS estimator of lobster counts
regressed on treatment. Use the function \texttt{summ()} from the
\href{https://jtools.jacob-long.com/}{\texttt{jtools}} package to print
the OLS output

\textbf{b.} Interpret the intercept \& predictor coefficients \emph{in
your own words}. Use full sentences and write your interpretation of the
regression results to be as clear as possible to a non-academic
audience.

\begin{Shaded}
\begin{Highlighting}[]
\CommentTok{\# Define simple OLS model of treatment impact on lobster counts}
\NormalTok{m1\_ols }\OtherTok{\textless{}{-}} \FunctionTok{lm}\NormalTok{(}
\NormalTok{    counts }\SpecialCharTok{\textasciitilde{}}\NormalTok{ treat,}
    \AttributeTok{data =}\NormalTok{ spiny\_counts}
\NormalTok{)}

\CommentTok{\# Print the model output}
\FunctionTok{summ}\NormalTok{(m1\_ols, }\AttributeTok{model.fit =} \ConstantTok{FALSE}\NormalTok{) }
\end{Highlighting}
\end{Shaded}

\emph{The simple OLS model predicted that there is a 5\% increase in
lobster abundance at the treatment sites compared to the baseline.}
\textbf{c.} Check the model assumptions using the \texttt{check\_model}
function from the \texttt{performance} package

\begin{Shaded}
\begin{Highlighting}[]
\CommentTok{\# Check model}
\FunctionTok{check\_model}\NormalTok{(m1\_ols,  }\AttributeTok{check =} \StringTok{"qq"}\NormalTok{ )}
\end{Highlighting}
\end{Shaded}

\textbf{d.} Explain the results of the 4 diagnostic plots. Why are we
getting this result?

\emph{In OLS there is an assumption of normality of residuals. In the
case of the the qq plot, ideally we would like to see the points fall
along the line, since we are seeing a curve it means that the residuals
are not normally distributed, violating the assumption of normality.}

\begin{Shaded}
\begin{Highlighting}[]
\CommentTok{\# Check normality}
\FunctionTok{check\_model}\NormalTok{(m1\_ols, }\AttributeTok{check =} \StringTok{"normality"}\NormalTok{)}
\end{Highlighting}
\end{Shaded}

\emph{This is another way to check the normality of the residuals.
Ideally, we would see the distribution of the residuals (shaded in blue)
fall within the bounds of the normal curve. This plot shows a skewed
distribution of residuals that doesn't follow the normal curve, it
reinforces that the residuals are not normally distributed. Since there
are more negative residuals, this could mean that the model is
underestimating the values.}

\begin{Shaded}
\begin{Highlighting}[]
\CommentTok{\# Check homogeneity}
\FunctionTok{check\_model}\NormalTok{(m1\_ols, }\AttributeTok{check =} \StringTok{"homogeneity"}\NormalTok{)}
\end{Highlighting}
\end{Shaded}

\emph{This plot shows the homogeneity of variance. Ideally, the
reference line would be flat, which would mean that there is little
variance of the residuals across the fitted values. Since we are seeing
a curve it means that the variance is not constant, violating the
assumption of homogeneity of variance.}

\begin{Shaded}
\begin{Highlighting}[]
\CommentTok{\# Check pp check}
\FunctionTok{check\_model}\NormalTok{(m1\_ols, }\AttributeTok{check =} \StringTok{"pp\_check"}\NormalTok{)}
\end{Highlighting}
\end{Shaded}

\emph{The posterior predictive check shows how well the model-predicted
data fits the observed data. In this case, the observed data is
significantly higher than the model-predicted data, which means that the
model is underestimating the predictions. This could be for a number of
reasons but indicates that we should maybe choose a different model that
better fits the data and/or we may need to transform the data.}

\begin{center}\rule{0.5\linewidth}{0.5pt}\end{center}

Step 5: Fitting GLMs

\textbf{a.} Estimate a Poisson regression model using the \texttt{glm()}
function

\begin{Shaded}
\begin{Highlighting}[]
\CommentTok{\# {-}{-}{-}{-} Fit a Poisson regression model with glm {-}{-}{-}{-}}

\NormalTok{m2\_pois }\OtherTok{\textless{}{-}} \FunctionTok{glm}\NormalTok{(}
\NormalTok{    counts }\SpecialCharTok{\textasciitilde{}}\NormalTok{ treat,}
    \AttributeTok{data =}\NormalTok{ spiny\_counts,}
    \AttributeTok{family =} \FunctionTok{poisson}\NormalTok{(}\AttributeTok{link =} \StringTok{"log"}\NormalTok{)}
\NormalTok{)}

\CommentTok{\# Summarize model}
\FunctionTok{summ}\NormalTok{(m2\_pois, }\AttributeTok{model.fit =} \ConstantTok{FALSE}\NormalTok{)}

\CommentTok{\# Transform the coefficient for IRR to get percent change}
\FunctionTok{exp}\NormalTok{(}\FloatTok{0.21}\NormalTok{) }\SpecialCharTok{{-}} \DecValTok{1}
\end{Highlighting}
\end{Shaded}

\textbf{b.} Interpret the predictor coefficient in your own words. Use
full sentences and write your interpretation of the results to be as
clear as possible to a non-academic audience.

\emph{The poisson model predicted that there is a 23\% increase in
lobster abundance at the treatment sites compared to the control.}

\textbf{c.} Explain the statistical concept of dispersion and
overdispersion in the context of this model. \emph{For a poisson model,
there is an assumption that the mean and the variance are supposed to be
equal. Dispersion describes the spread of data around the mean; and
overdispersion is used to describe when the variance is greater than the
mean. If there is overdispersion then the poisson model will
underestimate the variability.}

\textbf{d.} Compare results with previous model, explain change in the
significance of the treatment effect

\emph{The poisson model predicted that there is a 32\% increase in
lobster abundance at the treatment sites, while the simple OLS model
only estimated a 5\% increase. This makes sense because the OLS model
had a lot of negative residuals, which indicates that it was
underestimating the predictions.}

\textbf{e.} Check the model assumptions. Explain results.

\begin{Shaded}
\begin{Highlighting}[]
\CommentTok{\# Check the model assumptions}
\FunctionTok{check\_model}\NormalTok{(m2\_pois)}
\end{Highlighting}
\end{Shaded}

\emph{Looking at the QQ plot, it indicates that the model violates the
assumption of normality, since the residuals are distributed in an ``S''
curve along the quantiles, rather than following along the line. The PP
check shows that in some areas the model is underestimating and others
it is significantly overestimating. This model fits the assumption of
homogeneity. }

\textbf{f.} Conduct tests for over-dispersion \& zero-inflation. Explain
results.

\begin{Shaded}
\begin{Highlighting}[]
\CommentTok{\# Check for overdispersion}
\FunctionTok{check\_overdispersion}\NormalTok{(m2\_pois)}
\end{Highlighting}
\end{Shaded}

\emph{There was overdispersion detected in this model which means that
the variance is greater than the mean; this indicates that there could
be other factors (an omitted variable) influencing count that around
included in the model.}

\begin{Shaded}
\begin{Highlighting}[]
\CommentTok{\# Check for zero inflation}
\FunctionTok{check\_zeroinflation}\NormalTok{(m2\_pois)}
\end{Highlighting}
\end{Shaded}

\emph{There was no observed zeros in the response variable, which means
that there are other reasons, such as environmental factors,
contributing to overdispersion.}

\textbf{g.} Fit a negative binomial model using the function glm.nb()
from the package \texttt{MASS} and check model diagnostics

\begin{Shaded}
\begin{Highlighting}[]
\CommentTok{\# {-}{-}{-}{-} Fit a negative binomial model with glm.nb {-}{-}{-}{-}}

\NormalTok{m3\_nb }\OtherTok{\textless{}{-}} \FunctionTok{glm.nb}\NormalTok{(}
\NormalTok{    counts }\SpecialCharTok{\textasciitilde{}}\NormalTok{ treat,}
    \AttributeTok{data =}\NormalTok{ spiny\_counts}
\NormalTok{)}

\CommentTok{\# Summarize model}
\FunctionTok{summ}\NormalTok{(m3\_nb, }\AttributeTok{model.fit =} \ConstantTok{FALSE}\NormalTok{)}

\CommentTok{\# transform the coefficient}
\FunctionTok{exp}\NormalTok{(}\FloatTok{0.21}\NormalTok{) }\SpecialCharTok{{-}} \DecValTok{1}
\end{Highlighting}
\end{Shaded}

\textbf{h.} In 1-2 sentences explain rationale for fitting this GLM
model. \emph{Binomial models are good to use when data indicates
overdispersion because it handles variation better than models like OLS
or poisson, like in the case of our lobster count dataset, because the
nb model does not have the assumption that the mean = variance.}

\textbf{i.} Interpret the treatment estimate result in your own words.
Compare with results from the previous model.

\emph{The negative binomial model predicted that there is a 32\%
increase in lobster abundance at the treatment sites which is the same
as what the poisson model predicted, while the simple OLS model only
estimated a 5.9\% increase.}

\begin{Shaded}
\begin{Highlighting}[]
\CommentTok{\# Check for overdispersion}
\FunctionTok{check\_overdispersion}\NormalTok{(m3\_nb)}
\end{Highlighting}
\end{Shaded}

\emph{There is no overdispersion detected in this model, because it
accounts for the variance better.}

\begin{Shaded}
\begin{Highlighting}[]
\CommentTok{\# Check for zero inflation}
\FunctionTok{check\_zeroinflation}\NormalTok{(m3\_nb)}
\end{Highlighting}
\end{Shaded}

\emph{There were no observed zeros in this model, which means we dont
have to worry about that issue with this model.}

\begin{Shaded}
\begin{Highlighting}[]
\CommentTok{\# Check posterior predictions}
\FunctionTok{check\_predictions}\NormalTok{(m3\_nb)}
\end{Highlighting}
\end{Shaded}

\emph{The pp check shows a much better fit for this model. The model
predicted intervals match closely with the observed data points, which
means that the model-predicted data fits the observed data well.}

\begin{Shaded}
\begin{Highlighting}[]
\CommentTok{\# Check model assumptions}
\FunctionTok{check\_model}\NormalTok{(m3\_nb)}
\end{Highlighting}
\end{Shaded}

\emph{The checks all match the ideal distributions well, which means
that the predicted data and observed values fit the assumptions of the
model well.}

\begin{center}\rule{0.5\linewidth}{0.5pt}\end{center}

Step 6: Compare models

\textbf{a.} Use the \texttt{export\_summ()} function from the
\texttt{jtools} package to look at the three regression models you fit
side-by-side.

\begin{Shaded}
\begin{Highlighting}[]
\CommentTok{\# Create a table comparing model results}
\FunctionTok{export\_summs}\NormalTok{(m1\_ols, m2\_pois, m3\_nb,}
             \AttributeTok{model.names =} \FunctionTok{c}\NormalTok{(}\StringTok{"OLS"}\NormalTok{,}\StringTok{"Poisson"}\NormalTok{, }\StringTok{"NB"}\NormalTok{),}
             \AttributeTok{statistics =} \StringTok{"none"}\NormalTok{)}
\end{Highlighting}
\end{Shaded}

\textbf{c.} Write a short paragraph comparing the results. Is the
treatment effect \texttt{robust} or stable across the model
specifications.

\emph{For all three models, there was a consistent and statistically
significant positive treatment affect. Both the negative binomial model
and the poisson model predicted that there is a 32\% increase in lobster
abundance at the treatment sites, while the simple OLS model only
estimated a 5\% increase. The poisson and negative binomial models
performed similarly, despite the presence of overdispersion, which
indicates that the positive treatment effect is robust.}

\begin{center}\rule{0.5\linewidth}{0.5pt}\end{center}

Step 7: Building intuition - fixed effects

\textbf{a.} Create new \texttt{df} with the \texttt{year} variable
converted to a factor

\textbf{b.} Run the following OLS model using \texttt{lm()}

\begin{itemize}
\tightlist
\item
  Use the following specification for the outcome \texttt{log(counts+1)}
\item
  Estimate fixed effects for \texttt{year}
\item
  Include an interaction term between variables \texttt{treat} and
  \texttt{year}
\end{itemize}

\begin{Shaded}
\begin{Highlighting}[]
\CommentTok{\# {-}{-}{-}{-} Try again using \textquotesingle{}year\textquotesingle{} as a factor {-}{-}{-}{-}}
\NormalTok{ff\_counts }\OtherTok{\textless{}{-}}\NormalTok{ spiny\_counts }\SpecialCharTok{\%\textgreater{}\%} 
    \FunctionTok{mutate}\NormalTok{(}\AttributeTok{year=}\FunctionTok{as\_factor}\NormalTok{(year))}
    
\NormalTok{m5\_fixedeffs }\OtherTok{\textless{}{-}} \FunctionTok{glm.nb}\NormalTok{(}
\NormalTok{    counts }\SpecialCharTok{\textasciitilde{}} 
\NormalTok{        treat }\SpecialCharTok{+}
\NormalTok{        year }\SpecialCharTok{+}
\NormalTok{        treat}\SpecialCharTok{*}\NormalTok{year,}
    \AttributeTok{data =}\NormalTok{ ff\_counts)}

\FunctionTok{summ}\NormalTok{(m5\_fixedeffs, }\AttributeTok{model.fit =} \ConstantTok{FALSE}\NormalTok{)}
\end{Highlighting}
\end{Shaded}

\textbf{c.} Take a look at the regression output. Each coefficient
provides a comparison or the difference in means for a specific
sub-group in the data. Informally, describe the what the model has
estimated at a conceptual level (NOTE: you do not have to interpret
coefficients individually)

\begin{Shaded}
\begin{Highlighting}[]
\CommentTok{\# Check model summary}
\FunctionTok{summ}\NormalTok{(m5\_fixedeffs, }\AttributeTok{model.fit =} \ConstantTok{FALSE}\NormalTok{)}
\end{Highlighting}
\end{Shaded}

\emph{This model is predicting how the effect of treatment on lobster
counts varies by year. For most years, treatment had a positive effect
on lobster abundance however. However, the ``main effect'' is negative.}

\textbf{d.} Explain why the main effect for treatment is negative? *Does
this result make sense?

\emph{This means that for the baseline year, 2012, treatment groups had
lower lobster counts than the control group. This means that, on
average, the treatment groups started with fewer lobsters which could
introduce challenges in comparing the treated and non-treated areas.}

\textbf{e.} Look at the model predictions: Use the
\texttt{interact\_plot()} function from package \texttt{interactions} to
plot mean predictions by year and treatment status.

\begin{Shaded}
\begin{Highlighting}[]
\CommentTok{\# {-}{-}{-}{-} Plot mean predictions by year and treatment site with interact\_plot {-}{-}{-}{-}}
\FunctionTok{interact\_plot}\NormalTok{(m5\_fixedeffs,}
              \AttributeTok{pred =}\NormalTok{ year,}
              \AttributeTok{modx =}\NormalTok{ treat,}
              \AttributeTok{outcome.scale =} \StringTok{"response"}\NormalTok{)}
\end{Highlighting}
\end{Shaded}

\textbf{f.} Re-evaluate your responses (c) and (b) above. \emph{It means
that the mpa and non-mpa groups started at different baselines.}

\textbf{g.} Using \texttt{ggplot()} create a plot in same style as the
previous \texttt{interaction\ plot}, but displaying the original scale
of the outcome variable (lobster counts). This type of plot is commonly
used to show how the treatment effect changes across discrete time
points (i.e., panel data).

The plot should have\ldots{} - \texttt{year} on the x-axis -
\texttt{counts} on the y-axis - \texttt{mpa} as the grouping variable

\begin{Shaded}
\begin{Highlighting}[]
\CommentTok{\# Hint 1: Group counts by \textasciigrave{}year\textasciigrave{} and \textasciigrave{}mpa\textasciigrave{} and calculate the \textasciigrave{}mean\_count\textasciigrave{}}
\CommentTok{\# Hint 2: Convert variable \textasciigrave{}year\textasciigrave{} to a factor}

\CommentTok{\# Group by year and mpa}
\NormalTok{plot\_counts }\OtherTok{\textless{}{-}}\NormalTok{ spiny\_counts }\SpecialCharTok{|\textgreater{}} 
    \FunctionTok{group\_by}\NormalTok{(year, mpa) }\SpecialCharTok{|\textgreater{}}
    
    \CommentTok{\# Calculate mean}
    \FunctionTok{summarize}\NormalTok{(}
        \AttributeTok{mean\_count =} \FunctionTok{mean}\NormalTok{(counts, }\AttributeTok{na.rm =} \ConstantTok{TRUE}\NormalTok{)}
\NormalTok{    ) }\SpecialCharTok{|\textgreater{}}
    
    \FunctionTok{ungroup}\NormalTok{()}

\CommentTok{\# Create a line plot of mean count by year}
\NormalTok{plot\_counts }\SpecialCharTok{|\textgreater{}} \FunctionTok{ggplot}\NormalTok{(}\FunctionTok{aes}\NormalTok{(}\AttributeTok{x =}\NormalTok{ year, }\AttributeTok{y =}\NormalTok{ mean\_count, }\AttributeTok{color =}\NormalTok{ mpa)) }\SpecialCharTok{+}
    \FunctionTok{geom\_line}\NormalTok{()}
\end{Highlighting}
\end{Shaded}

\begin{center}\rule{0.5\linewidth}{0.5pt}\end{center}

Step 8: Reconsider causal identification assumptions

\begin{enumerate}
\def\labelenumi{\alph{enumi}.}
\tightlist
\item
  Discuss whether you think \texttt{spillover\ effects} are likely in
  this research context (see Glossary of terms;
  \url{https://docs.google.com/document/d/1RIudsVcYhWGpqC-Uftk9UTz3PIq6stVyEpT44EPNgpE/edit?usp=sharing})
\end{enumerate}

\emph{The spillover effect is likely to be present in this research
setting, especially since there are no hard boundary lines around the
perimeter of the the study areas. There are environmental factors, some
that could be influenced by the treatment, that could influence
competition/ecology of the area influencing lobsters to migrate between
the treated and control sites, especially for treatment sites that are
located next to a control site.}

\begin{enumerate}
\def\labelenumi{\alph{enumi}.}
\setcounter{enumi}{1}
\tightlist
\item
  Explain why spillover is an issue for the identification of causal
  effects
\end{enumerate}

\emph{Spillover is an issue because it violates the assumption of no
interference between the control and treatment group for causal
inference.}

\begin{enumerate}
\def\labelenumi{\alph{enumi}.}
\setcounter{enumi}{2}
\tightlist
\item
  How does spillover relate to impact in this research setting?
\end{enumerate}

\emph{It is important to consider the spillover effect of our treatment
areas when designing our study, so that we can choose control
sites/methods that properly account for it.}

\begin{enumerate}
\def\labelenumi{\alph{enumi}.}
\setcounter{enumi}{3}
\item
  Discuss the following causal inference assumptions in the context of
  the MPA treatment effect estimator. Evaluate if each of the assumption
  are reasonable:

  \begin{enumerate}
  \def\labelenumii{\arabic{enumii})}
  \item
    SUTVA: Stable Unit Treatment Value assumption \emph{SUTVA requires
    no-interference, which is described above, and no hidden variations
    in the treatment.From what I found on
    \href{https://wildlife.ca.gov/Conservation/Marine/MPAs/Naples}{wildlife.ca},
    the two treatment sites have different fishing regulations; while
    Isla Vista is a ``No-Take'' zone, Naples allows recreational take of
    some pelagic fish and for the commercial take of kelp, which I think
    would violate the requirement for no variations in treatment (since
    we didn't account for it in our models). So I would argue that both
    of these assumptions are not reasonable as is.}
  \item
    Excludability assumption \emph{Since the sites are all located in
    the greater Santa Barbara region, dispersed broadly across the coast
    and likely have different baselines in variables such as proximity
    to pollution, habitat disturbance, among other environmental/human
    factors that may introduce challenges in comparing the sites. I
    don't think that excludability is a reasonable assumption as is.}
    ------------------------------------------------------------------------
  \end{enumerate}
\end{enumerate}

\section{EXTRA CREDIT}\label{extra-credit}

\begin{quote}
Use the recent lobster abundance data with observations collected up
until 2024 (\texttt{lobster\_sbchannel\_24.csv}) to run an analysis
evaluating the effect of MPA status on lobster counts using the same
focal variables.
\end{quote}

\begin{enumerate}
\def\labelenumi{\alph{enumi}.}
\tightlist
\item
  Create a new script for the analysis on the updated data
\item
  Run at least 3 regression models \& assess model diagnostics
\item
  Compare and contrast results with the analysis from the 2012-2018 data
  sample (\textasciitilde{} 2 paragraphs)
\end{enumerate}

\begin{center}\rule{0.5\linewidth}{0.5pt}\end{center}

\includegraphics{figures/spiny1.png}

\end{document}
